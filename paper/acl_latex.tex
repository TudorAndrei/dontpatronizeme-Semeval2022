% This must be in the first 5 lines to tell arXiv to use pdfLaTeX, which is strongly recommended.
\pdfoutput=1
% In particular, the hyperref package requires pdfLaTeX in order to break URLs across lines.

\documentclass[11pt]{article}

% Remove the "review" option to generate the final version.
\usepackage[review]{acl}

% Standard package includes
\usepackage{times}
\usepackage{latexsym}

% For proper rendering and hyphenation of words containing Latin characters (including in bib files)
\usepackage[T1]{fontenc}
% For Vietnamese characters
% \usepackage[T5]{fontenc}
% See https://www.latex-project.org/help/documentation/encguide.pdf for other character sets

% This assumes your files are encoded as UTF8
\usepackage[utf8]{inputenc}

% This is not strictly necessary, and may be commented out,
% but it will improve the layout of the manuscript,
% and will typically save some space.
\usepackage{microtype}

% If the title and author information does not fit in the area allocated, uncomment the following
%
%\setlength\titlebox{<dim>}
%
% and set <dim> to something 5cm or larger.

\title{Instructions for *ACL Proceedings}

% Author information can be set in various styles:
% For several authors from the same institution:
% \author{Author 1 \and ... \and Author n \\
%         Address line \\ ... \\ Address line}
% if the names do not fit well on one line use
%         Author 1 \\ {\bf Author 2} \\ ... \\ {\bf Author n} \\
% For authors from different institutions:
% \author{Author 1 \\ Address line \\  ... \\ Address line
%         \And  ... \And
%         Author n \\ Address line \\ ... \\ Address line}
% To start a seperate ``row'' of authors use \AND, as in
% \author{Author 1 \\ Address line \\  ... \\ Address line
%         \AND
%         Author 2 \\ Address line \\ ... \\ Address line \And
%         Author 3 \\ Address line \\ ... \\ Address line}

\author{Person 1\and Person 3 \and Person 5 \and Person 2\\
        {tone}@edu.ro}
% \author{First Author \\
%   Affiliation / Address line 1 \\
%   Affiliation / Address line 2 \\
%   Affiliation / Address line 3 \\
%   \texttt{email@domain} \\\And
%   Second Author \\
%   Affiliation / Address line 1 \\
%   Affiliation / Address line 2 \\
%   Affiliation / Address line 3 \\
%   \texttt{email@domain} \\}

\begin{document}
\maketitle
\begin{abstract}
This document is a supplement to the general instructions for *ACL authors. It contains instructions for using the \LaTeX{} style files for ACL conferences.
The document itself conforms to its own specifications, and is therefore an example of what your manuscript should look like.
These instructions should be used both for papers submitted for review and for final versions of accepted papers.
\end{abstract}

\section{Introduction}

These instructions are for authors submitting papers to *ACL conferences using \LaTeX. They are not self-contained. All authors must follow the general instructions for *ACL proceedings,\footnote{\url{http://acl-org.github.io/ACLPUB/formatting.html}} and this document contains additional instructions for the \LaTeX{} style files.

The templates include the \LaTeX{} source of this document (\texttt{acl.tex}),
the \LaTeX{} style file used to format it (\texttt{acl.sty}),
an ACL bibliography style (\texttt{acl\_natbib.bst}),
an example bibliography (\texttt{custom.bib}),
and the bibliography for the ACL Anthology (\texttt{anthology.bib}).

\section{Background}



In your own words, summarize important details about the task setup: kind of input and output (give an example if possible); what datasets were used, including language, genre, and size. If there were multiple tracks, say which you participated in.

Here or in other sections, cite related work that will help the reader to understand your contribution and what aspects of it are novel.

\section{System Overview}


Key algorithms and modeling decisions in your system; resources used beyond the provided training data; challenging aspects of the task and how your system addresses them. This may require multiple pages and several subsections, and should allow the reader to mostly reimplement your system’s algorithms.

Use equations and pseudocode if they help convey your original design decisions, as well as explaining them in English. If you are using a widely popular model/algorithm like logistic regression, an LSTM, or stochastic gradient descent, a citation will suffice—you do not need to spell out all the mathematical details.

Give an example if possible to describe concretely the stages of your algorithm.

If you have multiple systems/configurations, delineate them clearly.

This is likely to be the longest section of your pap

\section{Experimental setup}



How data splits (train/dev/test) are used.

Key details about preprocessing, hyperparameter tuning, etc. that a reader would need to know to replicate your experiments. If space is limited, some of the details can go in an Appendix.

External tools/libraries used, preferably with version number and URL in a footnote.

Summarize the evaluation measures used in the task.

You do not need to devote much—if any—space to discussing the organization of your code or file formats.

\section{Results}



Main quantitative findings: How well did your system perform at the task according to official metrics? How does it rank in the competition?

Quantitative analysis: Ablations or other comparisons of different design decisions to better understand what works best. Indicate which data split is used for the analyses (e.g. in table captions). If you modify your system subsequent to the official submission, clearly indicate which results are from the modified system.

Error analysis: Look at some of your system predictions to get a feel for the kinds of mistakes it makes. If appropriate to the task, consider including a confusion matrix or other analysis of error subtypes—you may need to manually tag a small sample for this.
\cite{Aho:72}

\section{Conclusion}

\subsection{References}

\bibliography{custom}

\appendix

\section{Example Appendix}
\label{sec:appendix}

This is an appendix.

\end{document}
